\documentclass[a4paper,twoside]{report}
\usepackage[utf8]{inputenc}
% \usepackage[a4paper, width=150mm, top=25mm, bottom=25mm, bindingoffset=6mm]{geometry}
\usepackage{fancyhdr}
\pagestyle{fancy}
 \fancyhead{}
 \fancyhead[LO,LE]{\leftmark}
 \fancyfoot{}
 \fancyfoot[CO,CE]{\thepage}
 \renewcommand{\headrulewidth}{0.2pt}
 \renewcommand{\footrulewidth}{0.2pt}
 \setlength{\headheight}{14.5pt}


% \usepackage{epsfig,epic,eepic,units}
\usepackage{hyperref} %allows to add hyperlinks and cross reference in-text
\usepackage{textcomp} %this is for symbols such as copyright etc
\usepackage{url} %for adding urls obvs so they appear always in one line without breaks
\usepackage{longtable} %allows you to make tables of two or more pages
\usepackage{mathrsfs} %maths fonts, might not need
\usepackage{array} %extends array and tabular environments
\usepackage{multirow} %for tables again
\usepackage{bigstrut} %tables
\usepackage{amssymb} %adding arrows tbh for chemistry you can just use chemmacros \begin{reaction} environment
\usepackage{amsmath} %for writing maths
\usepackage{adjustbox}%supplement to graphics package
\usepackage{graphicx}
\usepackage{lscape} %allows to add certain pages in landscape
\usepackage[nottoc]{tocbibind} %this adds bibliography to your ToC
\usepackage{textgreek} %for upright greek letters
\usepackage{siunitx} %you can declare the units you want as necessary read documentation
\usepackage{biblatex}
\usepackage[
    asymmetric,
    a4paper,
    left=3cm,
    right=2cm,
    top=2.75cm,
    bottom=3cm,
    footskip=20pt,
    twoside]{geometry}

    \usepackage{tikz}
\usetikzlibrary {automata,positioning}
\usepackage{enumitem}
\usepackage{pifont}

% add commands
\usepackage{commands}


\addbibresource{refs.bib}

\begin{document}
\sloppy

\section{Different Behaviors for the Holes}
% For different behaviors to be observable. Let $sentTotal$ and $ackedTotal$ be number of sent frames and acknowledged frames in total.


We build an OA serving as a prototype for connection-oriented communication protocols. Let $N$ be the number of data units to be sent, whose index ranges from $0$ to $N-1$. There are two holes, namely Sender and Receiver. The Sender invokes actions \texttt{send, receive} and \texttt{process}, corresponding to sending a data unit, receiving an acknowledgement and doing an internal work, such as waiting until timeout. The Receiver also invokes actions \texttt{send}, \texttt{receive} and \texttt{process}, but here it sends an acknowledgement and receives a data unit. A data unit or an acknowledgement may also be lost during transmission. From the environment's perspective, a data unit can be in one of the following situation

\begin{itemize}
	\item it needs to be sent,
	\item it is on fly i.e. it was sent by the Sender but not arrives at the Receiver yet,
	\item it has been received by the Receiver, or
	\item it has been sent and acknowledged successfully.
\end{itemize}

Hence we use four variables of type lists $D_{\text{tosend}}$, $D_{\text{onfly}}$, $D_{\text{received}}$ and $D_{\text{done}}$ to store the indices of the data units in these situations. For an acknowledgement, it can also by on fly, which is store in $A_\text{onfly}$. We do not need to store received acknowledgements because the Sender  processes an acknowledgement immediately when receiving it. The initial values are $D_{\text{tosend}} = [0,\ldots, N-1]$ and $D_{\text{onfly}}=D_{\text{received}}=D_{\text{done}}=A_{\text{onfly}}=[]$. We use two operation on a list $L$

\begin{itemize}
	\item $L.\text{push}(i)$: appending $i$ to the list, and
	\item $L.\text{pop}(i)$: removing $i$ in its first appearance from the list or return the old list if there is no $i$
\end{itemize}

and a function $\min(L)$ to find the minimal element in the list. The transitions are given in the following table.

\begin{table}[h]
	\centering
	\begin{tabular}{|c|c|c|c|c|c|c|}
		\hline
		N.O. & Source & Target & Action                  & Hole actions                                  & Guard                         & Reassignment                                                                                                                                         \\
		\hline
		1    & $q_0$  & $q_0$  & $\text{sendData}(i)$    & $\text{Sender}\mapsto \texttt{send}(i)$       & $i = \min(D_{\text{tosend}})$ & $D_{\text{onfly}}.\text{push}(i)$                                                                                                                    \\
		\hline
		2    & $q_0$  & $q_0$  & $\text{receiveAck}(i)$  & $\text{Sender}\mapsto \texttt{receive}(i)$    & $i \in A_{\text{onfly}}$      & \begin{tabular}{@{}c@{}}$A_{\text{onfly}}.\text{pop}(i)$ \\ $D_{\text{tosend}}.\text{pop}(i)$\end{tabular}  \\
		\hline
		3    & $q_0$  & $q_0$  & $\tau$                  & $\text{Sender}\mapsto \texttt{process}(opts)$ & True                          &                                                                                                                                                      \\
		\hline
		4    & $q_0$  & $q_0$  & $\text{receiveData}(i)$ & $\text{Receiver}\mapsto \texttt{receive}(i)$  & $i\in D_{\text{onfly}}$       & \begin{tabular}{@{}c@{}}$D_{\text{onfly}}.\text{pop}(i)$ \\ $D_{\text{received}}.\text{push}(i)$\end{tabular} \\
		\hline
		5    & $q_0$  & $q_0$  & $\text{sendAck}(i)$     & $\text{Receiver}\mapsto \texttt{send}(i)$     & $i\in D_{\text{received}}$    & $D_{\text{received}}.\text{pop}(i)$                                                                                                                  \\
		\hline

		6    & $q_0$  & $q_0$  & $\tau$                  & $\text{Sender}\mapsto \texttt{process}(opts)$ & True                          &                                                                                                                                                      \\
		\hline
		7    & $q_0$  & $q_0$  & loseData$(i)$           &                                               & $i\in D_\text{onfly}$         & $D_{\text{onfly}}.\text{pop}(i)$                                                                                                                     \\
		\hline
		8    & $q_0$  & $q_0$  & loseAck$(i)$            &                                               & $i\in A_\text{onfly}$         & $A_{\text{onfly}}.\text{pop}(i)$                                                                                                                     \\
		\hline
	\end{tabular}
	\caption{Transition Table for Protocol OA}
	\label{tab:transitions}
\end{table}

An protocol implementing the Sender and Receiver is correct if given the initial assignments of the lists, the composed OA always arrives at a terminal state with $D_\text{done} = [1,\ldots,N]$.

Consider and example. Let $D_{\text{tosend}} = [0, 1, 2]$ and $D_{\text{onfly}}=D_{\text{received}}=D_{\text{done}}=A_{\text{onfly}}=[]$. Denote by $(D_{\text{tosend}}, D_{\text{onfly}}, D_{\text{received}}, A_{\text{onfly}})$ each configuration of the OA. One possible run is
\begin{align*}
	([0,1,2], [], [], [], []) \xrightarrow{\text{sendData}(0)} ([1,2], [0], [], [], [])
\end{align*}

% \printbibliography

\end{document}
