\section{Open Automaton for TCP Sliding Window}

\section{Open Automaton for Data Link Go-back-\texorpdfstring{$N$}{N}}
\subsection{Specification}

The sender maintains a circular buffer $B$ of size $N$, where $N$ is the window size. Let $i$ and $j$ denote the indices of the oldest unacknowledged frame and the next frame to send, respectively. Both range over $[0, N-1]$ and are interpreted modulo $N$. At the beginning, $i = j = 0$. The sender may transmit a frame $B[j]$ if the window is not full, i.e., $j \neq (i - 1) \bmod N$. After transmission, $j$ is incremented modulo $N$. The receiver maintains a single expected frame index, initially set to $0$. It accepts a frame if and only if its index matches the expected $i$, acknowledges it, and increments $i$ modulo $N$. All other frames are discarded. Upon receiving an acknowledgment for frame $i$, the sender removes the acknowledged frame and increments $i$ modulo $N$. If an acknowledgment for frame $i$ is not received within a timeout interval $T$ after it was sent, the sender retransmits all frames currently in the window, starting from frame $i$ up to (but not including) frame $j$.

\subsection{Open Automaton}
From the specification, given that Sender and Receiver as holes, two variables $i$ and $j$ are required. The Sender emits observable actions \texttt{send}, \texttt{wait}, \texttt{resend} and an unobservable action \texttt{getFrame}, which fetches a packet from the network layer and converts it to a frame. The Receiver emits an observable action \texttt{ack} and an unobservable action \texttt{dispatchFrame}, which sends a frame to the network layer. In addition, let $t\in[0,T]$ be the variable for waited time. We assume that for each time \texttt{wait} is emitted, $t$ in incremented.

The open automation has a single state $q_0$ and the transitions are

$$
  \begin{aligned}
    s_1 & = \dfrac{\{\text{Sender}\mapsto \texttt{getFrame}\}(\text{True})\{\}}{q_0\xrightarrow{\tau}q_0}                                \\
    s_2 & = \dfrac{\{\text{Sender}\mapsto \texttt{send}(j)\}((j+1)\% N \neq i)\{j:= (j + 1)\%N\}}{q_0\xrightarrow{\text{frameSent}}q_0}  \\
    s_3 & = \dfrac{\{\text{Sender}\mapsto \texttt{wait}\}((j+1)\% N = i \wedge t < T)\{t:= t + 1\}}{q_0\xrightarrow{\tau}q_0}            \\
    s_4 & = \dfrac{\{\text{Sender}\mapsto \texttt{resend}(i)\}(t = T)\{t:= 0\}}{q_0\xrightarrow{\text{frameResent}}q_0}                  \\
    t_1 & = \dfrac{\{\text{Receiver}\mapsto \texttt{ack}(i)\}(i \neq j)\{i:= (i + 1)\%N; t:= 0\}}{q_0\xrightarrow{\text{frameAcked}}q_0} \\
    t_2 & = \dfrac{\{\text{Receiver}\mapsto \texttt{dispatchFrame}\}(\text{True})\{\}}{q_0\xrightarrow{\tau}q_0}                         \\
  \end{aligned}
$$

\begin{center}
  \begin{tikzpicture}[shorten >=1pt,node distance=2cm,on grid,auto]

    \node[]               (init_assign)
    {};
    \node[state, accepting] (q_0) [right=4cm of init_assign] {$q_0$};


    \path[->]
    (init_assign) edge node {$\begin{Bmatrix}
            i,j := 0 \\
            t := 0   \\
          \end{Bmatrix}$} (q_0)
    (q_0) edge [loop, in=-140, out=-110, looseness=10] node[pos=0.5, anchor=north] {$s_1$} (q_0)
    (q_0) edge [loop, in=-60, out=-90, looseness=10] node[pos=0.5, anchor=north] {$s_2$} (q_0)
    (q_0) edge [loop, in=-10, out=-40, looseness=10] node[pos=0.5, anchor=north] {$s_3$} (q_0)
    (q_0) edge [loop, in=40, out=10, looseness=10] node[pos=0.5, anchor=south] {$s_4$} (q_0)
    (q_0) edge [loop, in=90, out=60, looseness=10] node[pos=0.5, anchor=south] {$t_1$} (q_0)
    (q_0) edge [loop, in=140, out=110, looseness=10] node[pos=0.5, anchor=south] {$t_2$} (q_0)
    ;

  \end{tikzpicture}
\end{center}

\section{Open Automaton for Data Link Selective Repeat}
\subsection{Specification}
Selective repeat protocol allows the sender to only resend the oldest unacknowledged frame. In addition to a buffer $B$ and two indices $i$ and $j$, the sender needs an acknowledged array $Acked$ whose values are boolean. Every time the receiver receives a frame correctly, it sends an acknowledgment. When an acknowledgment is received, the sender either update the acknowledgment status of the corresponding frame, or when the frame is at index $i$, the sender rotates the sliding window and set the acknowledgment status for this frame to False.

\subsection{Open Automaton}
We need an additional array $Acked$ for acknowledgment status. Let the function $oldestUnacked$ return the oldest unacknowledged frame index. The transitions are

$$
  \begin{aligned}
    s_1 & = \dfrac{\{\text{Sender}\mapsto \texttt{getFrame}\}(\text{True})\{\}}{q_0\xrightarrow{\tau}q_0}                                                          \\
    s_2 & = \dfrac{\{\text{Sender}\mapsto \texttt{send}(j)\}((j+1)\% N \neq i)\{j:= (j + 1)\%N\}}{q_0\xrightarrow{\text{frameSent}}q_0}                            \\
    s_3 & = \dfrac{\{\text{Sender}\mapsto \texttt{wait}\}((j+1)\% N = i \wedge t < T)\{t:= t + 1\}}{q_0\xrightarrow{\tau}q_0}                                      \\
    s_4 & = \dfrac{\{\text{Sender}\mapsto \texttt{resend}(oldestUnacked())\}(t = T)\{t:= 0\}}{q_0\xrightarrow{\text{frameResent}}q_0}                              \\
    t_1 & = \dfrac{\{\text{Receiver}\mapsto \texttt{ack}(i)\}(i \neq j)\{i:= (i + 1)\%N; t:= 0; Acked[i] := \text{False}\}}{q_0\xrightarrow{\text{frameAcked}}q_0} \\
    t_2 & = \dfrac{\{\text{Receiver}\mapsto \texttt{ack}(k)\}(i < k \ne j)\{Acked[k] := \text{True}\}}{q_0\xrightarrow{\text{frameAcked}}q_0}                      \\
    t_3 & = \dfrac{\{\text{Receiver}\mapsto \texttt{dispatchFrame}\}(\text{True})\{\}}{q_0\xrightarrow{\tau}q_0}                                                   \\
  \end{aligned}
$$


\begin{center}
  \begin{tikzpicture}[shorten >=1pt,node distance=2cm,on grid,auto]

    \node[]               (init_assign)
    {};
    \node[state, accepting] (q_0) [right=4cm of init_assign] {$q_0$};


    \path[->]
    (init_assign) edge node {$\begin{Bmatrix}
            i,j := 0 \\
            t := 0   \\
            Acked := [\text{False}]
          \end{Bmatrix}$} (q_0)

    (q_0) edge [loop, in=-60, out=-90, looseness=10] node[pos=0.5, anchor=north] {$s_1,s_2,s_3,s_4$} (q_0)

    (q_0) edge [loop, in=90, out=60, looseness=10] node[pos=0.5, anchor=south] {$t_1, t_2,t_3$} (q_0)
    ;

  \end{tikzpicture}
\end{center}