\section{Binary Session Types}

\begin{frame}{Type Syntax}
  One version is \footnote{Dardha, Ornela, Elena Giachino, and Davide Sangiorgi. "Session types revisited." Proceedings of the 14th symposium on Principles and practice of declarative programming. 2012.}
  \begin{center}
    \begin{minipage}{0.48\textwidth}
      $$\begin{array}{lclr}
          T & ::= & S      & \text{(session type)}     \\
            & |   & \cdots & \text{(other constructs)}
        \end{array}$$
    \end{minipage}
    \hfill
    \begin{minipage}{0.48\textwidth}
      $$\begin{array}{lclr}
          S & ::= & \texttt{end}                  & \text{(termination)} \\
            & |   & ?T.S                          & \text{(receive)}     \\
            & |   & !T.S                          & \text{(send)}        \\
            & |   & \&\{\ell_i:S_i\}_{i\in I}     & \text{(branch)}      \\
            & |   & \oplus\{\ell_i:S_i\}_{i\in I} & \text{(select)}      \\
        \end{array}$$
    \end{minipage}
  \end{center}
  \textbf{Note:} The suffix of a type is always \texttt{end}, a branch or a select.
\end{frame}

\begin{frame}{Type Syntax}
  Example: Mathematical server type that receives two integers and sends a result whether they are equal, then ends the session
  $$?\texttt{Int}.?\texttt{Int}.!\texttt{Bool}.\texttt{end}.$$
  But there are also weird types like
  \begin{align*}
    T \longrightarrow S \longrightarrow ?T.S \longrightarrow ?T.\texttt{end} \longrightarrow ?S.\texttt{end} \\
    \longrightarrow ??T.S.\texttt{end} \longrightarrow ??T.\texttt{end}.\texttt{end} \longrightarrow ??\texttt{Int}.\texttt{end}.\texttt{end}
  \end{align*}
  There may be no problem in practice.

  Qualifiers \texttt{lin} and \texttt{un} are invented to solve this.
\end{frame}

\begin{frame}{Type Syntax}
  Alternative \footnote{\href{https://stanford-cs242.github.io/f19/lectures/09-1-session-types}{CS242 Lecture 9.1}}
  \begin{center}
    \begin{minipage}{0.48\textwidth}
      $$\begin{array}{lclr}
          T & ::= & S & \text{(session type)} \\
            & |   & D & \text{(data type)}
        \end{array}$$
    \end{minipage}
    \hfill
    \begin{minipage}{0.48\textwidth}
      $$\begin{array}{lclr}
          S & ::= & \texttt{end}                  & \text{(termination)}            \\
            & |   & ?D.S                          & \text{(receive)}                \\
            & |   & !D.S                          & \text{(send)}                   \\
            & |   & \&\{\ell_i:S_i\}_{i\in I}     & \text{(branch)}                 \\
            & |   & \oplus\{\ell_i:S_i\}_{i\in I} & \text{(select)}                 \\
            & |   & \mu\alpha.S                   & \text{(recursive session type)} \\
            & |   & \alpha                        & \text{(session type variable)}
        \end{array}$$
    \end{minipage}
  \end{center}
\end{frame}

\begin{frame}{Type Syntax}
  Example: Session type for ATM protocol from \textit{ATM's point of view}
  \begin{align*}
    S_\text{ATM} = \,\,?\text{String}.\oplus
    \begin{cases}
      \text{auth:} & \mu\alpha.\&
      \begin{cases}
        \text{deposit:}   & ?Int.!Int.\alpha \\
        \text{withdraw:}  & ?Int.\oplus
        \begin{cases}
          \text{ok:}  & \alpha       \\
          \text{err:} & \texttt{end} \\
        \end{cases}           \\
        \text{terminate:} & \texttt{end}
      \end{cases} \\
      \text{err:}  & \texttt{end}
    \end{cases}
  \end{align*}

\end{frame}

\begin{frame}
  The client uses the type $S_{\text{Clt}}$ dual to ensure agreement. Rules for duality are
  \begin{align*}
    \overline{\texttt{end}}                  & = \texttt{end}                             \\
    \overline{?D.S}                          & = !D.\overline{S}                          \\
    \overline{!D.S}                          & = ?D.\overline{S}                          \\
    \overline{\&\{\ell_i:S_i\}_{i\in I}}     & = \oplus\{\ell_i:\overline{S_i}\}_{i\in I} \\
    \overline{\oplus\{\ell_i:S_i\}_{i\in I}} & = \&\{\ell_i:\overline{S_i}\}_{i\in I}     \\
    \overline{\mu\alpha.S}                   & = \mu\alpha.\overline{S}
  \end{align*}

  \textbf{Note:} Having known the type for one side, we know the type for the other side.
\end{frame}

\begin{frame}{Process syntax}
  Major notions
  \begin{itemize}
    \item Input, output, branching, selection
    \item Channel: using one polarized channel variable or two channel variables (we consider the latter)
    \item Composition (two processes running at the same time)
    \item Inaction (process stops)
  \end{itemize}

  Assumptions: a set of variables ranged by $x,y,\ldots$, possibly with subscripts and a set expressions over these variables.
\end{frame}

\begin{frame}{Process syntax}
  \begin{center}
    \begin{minipage}{0.48\textwidth}
      $$\begin{array}{lclr}
          P,Q & ::= & x!\langle v\rangle.P                      & \text{(output)}    \\
              & |   & x!(y).P                                   & \text{(input)}     \\
              & |   & x \triangleright \{\ell_i: P_i\}_{i\in I} & \text{(branching)} \\
              & |   & x \triangleleft \ell_j.P                  & \text{(selection)} \\
        \end{array}$$
    \end{minipage}
    \hfill
    \begin{minipage}{0.48\textwidth}
      $$\begin{array}{lclr}
           & | & \mathbf{0}                                   & \text{(inaction)}         \\
           & | & P | Q                                        & \text{(composition)}      \\
           & | & (\bm{\nu} xy)P                               & \text{(channel creation)} \\
           & | & \text{if } v \text{ then } P \text{ else } Q & \text{(conditional)}
        \end{array}$$
    \end{minipage}
  \end{center}

  % \textbf{Note:} What if a process need to select conditionally? We can add to the syntax, for example
  % $$\text{case}(e)\{v_i : P_i\}_{i\in I}, \text{or}$$
  % $$\text{if } e \text{ then }P\text{ else }Q.$$
  % then redefine compatibility and calculus.
\end{frame}

\begin{frame}{Typing rules}
  Typing environment $\Gamma$ is a map from identifiers to types. By convention, $\Gamma, x: T$ means overwrite $x$ with type $T$.

  Basic typing rules \footnote{Honda, Kohei, Vasco T. Vasconcelos, and Makoto Kubo. "Language primitives and type discipline for structured communication-based programming." European Symposium on Programming. Berlin, Heidelberg: Springer Berlin Heidelberg, 1998.}
  \begin{center}
    \begin{minipage}{0.48\textwidth}
      $$\dfrac{\Gamma, x: T_2 \vdash P}{\Gamma, y:T_1, x:!T_1.T_2 \vdash x!\langle y\rangle.P}$$
      Read: If $P$ uses $x$ with type $T_2$, then we can use $x$ with type $!T_1.T_2$ to send a message bound in $y$ of type $T_1$, then safely continue as $P$.
    \end{minipage}
    \hfill
    \begin{minipage}{0.48\textwidth}
      $$\dfrac{\Gamma, y: T_1, x: T_2 \vdash P}{\Gamma, x:?T_1.T_2 \vdash x?(y).P}$$
      Read: If $P$ uses $y$ with type $T_1$ and $x$ with type $T_2$, then we can use $x$ of type $?T_1.T_2$ to receive a message, bind it into $y$, then safely continue as $P$.
    \end{minipage}
  \end{center}
\end{frame}

\begin{frame}{Typing rules}
  Or we can distinguish between session types and data types.
  \begin{center}
    \begin{minipage}{0.48\textwidth}
      $$\dfrac{\Gamma, x: S \vdash P}{\Gamma, y:D, x:!D.S \vdash x!\langle y\rangle.P}$$
    \end{minipage}
    \hfill
    \begin{minipage}{0.48\textwidth}
      $$\dfrac{\Gamma, y: D, x: S \vdash P}{\Gamma, x:?D.S \vdash x?(y).P}$$
    \end{minipage}
  \end{center}
\end{frame}

\begin{frame}{Process calculus}
  Rules to ``run'' processes, defined using a reduction relation $\longrightarrow$. There is also a structural congruence relation $\equiv$.

  Recall types $S_\text{ATM}$ and $S_\text{Clt}$. Consider two processes

  $$clt(x: S_\text{Clt}) = x!\langle ``abcde'' \rangle. x \triangleright
    \begin{cases}
      \text{auth:} x\triangleleft \text{deposit}. x!50.x?a.x\triangleleft \text{terminate}.\texttt{end} \\
      \text{err:} \texttt{end}
    \end{cases}$$

  Suppose that $d$ saves the current balance.

  $$atm(y: S_\text{ATM}) = y?(b). \text{if } (b = ``abcde'') y \triangleleft \text{auth}. atm_{auth}  \text{ else } y \triangleleft \text{err}.\texttt{end}$$
  $$atm_{auth}(y) = y \triangleright
    \begin{cases}
      \text{deposit:} y?(c).y!\langle d+c \rangle.atm_{auth}(y)                                                                                \\
      \text{withdraw:} y?(c). \text{ if } (d > c) y\triangleleft \text{ok}.atm_{auth}(y) \text{ else } y \triangleleft \text{err}.\texttt{end} \\
      \text{terminate:} \texttt{end}
    \end{cases}$$
\end{frame}

\begin{frame}{Process calculus}
  \begin{align*}
    (\nu xy)\bigl(clt(x: S_\text{Clt})                                                                  & | atm(y: S_\text{ATM})\bigr)                                                                                                             \\
    (\nu xy)(x!\langle ``abcde'' \rangle. x \triangleright \ldots                                       & \big| y?(b). \text{if } (b = ``abcde'') y \triangleleft \text{auth}. atm_{auth}  \text{ else } y \triangleleft \text{err}.\texttt{end})  \\
    (\nu xy)(x \triangleright \ldots                                                                    & \big| \text{if } (``abcde'' = ``abcde'') y \triangleleft \text{auth}. atm_{auth}  \text{ else } y \triangleleft \text{err}.\texttt{end}) \\
    (\nu xy)\Biggl(x \triangleright  \begin{cases}
                                       \text{auth:} \ldots \\
                                       \text{err:} \texttt{end}
                                     \end{cases}                                                      & \Bigg|  y \triangleleft \text{auth}. atm_{auth}\Biggr)                                                                                     \\
    (\nu xy)\Biggl(x\triangleleft \text{deposit}. x!50.x?a.x\triangleleft \text{terminate}.\texttt{end} & | y \triangleright
    \begin{cases}
      \text{deposit:} y?(c).y!\langle d+c \rangle.atm_{auth}(y) \\
      \ldots
    \end{cases}\Biggr)                                                                                                                                                                                      \\
    (\nu xy)(x!50.x?a.x\triangleleft \text{terminate}.\texttt{end}                                      & | y?(c).y!\langle d+c \rangle.atm_{auth}(y))                                                                                             \\
    (\nu xy)(x?a.x\triangleleft \text{terminate}.\texttt{end}                                           & | y!\langle d+50 \rangle.atm_{auth}(y))                                                                                                  \\
    (\nu xy)(x\triangleleft \text{terminate}.\texttt{end}                                               & | atm_{auth}(y))                                                                                                                         \\
    (\nu xy)(\texttt{end}                                                                               & | \texttt{end})                                                                                                                          \\
  \end{align*}
\end{frame}

\begin{frame}{Process calculus}
  Selecting is basically sending a label. There are formalisms that all transmissions are labeled.

  There can be different notions for successful communication: progress, both sides end, only client ends.
\end{frame}